% %Resumen % %

\section{Resumen}
El trabajo colaborativo es una actividad compleja debido a que un grupo de personas tienen que unir sus esfuerzos para poder llevar alcanzar un objetivo realizando tareas que en ocasiones dependen de otras realizadas por otros integrantes del grupo. El trabajo colaborativo asistido por computadora (CSCW) \footnote{CSCW por sus siglas en ingl\'es(\textit{Comupter Supported Collaborative Work})} es el \'area que estudia los sistemas computacionales que ayudan a los grupos de trabajo a mejorar la coordinaci\'on, cooperaci\'on y comunicaci\'on que hay entre los integrantes del grupo de trabajo.

Entre los sistemas que estudia el CSCW est\'an los Groupware que son sistemas que fungen como medio para que los usuarios interact\'uen entre si y puedan llevar a cabo tareas en conjunto. Para facilitar la interacci\'on de estos usuarios con el sistema y con otros usuarios necesitan tener un grado aceptable de consciencia de la situaci\'on que los rodea a ellos como individuos, de los miembros del grupo del que forman parte y al estado de las tareas que se est\'an llevando a cabo. Debido a esto surge la necesidad de dotar a los sistemas con el conocimiento necesario para poder proporcionar tales niveles de conscienca al usuario a partir de esta necesidad se desarrollan sistemas conscientes del contexto los cuales tienen conocimiento de la situaci\'on del ambiente en el que operan, as\'i, los Sistemas Groupware Conscientes del Contexto \footnote{CAGS por sus siglas en ingl\'es \textit{Context Aware Groupware Sistems}} tienen ambas caracter\'isticas antes descritas: apoyan el trabajo colaborativo de un grupo de usuarios y adem\'as conoce el contexto que los rodea y opera en consecuencia a esta informaci\'on, para eso se crean arquitecturas conscientes del contexto que puedan ser integradas a un groupware y que entre ambos sistemas puedan funcionar como un CAGS.

Actualmente la mayor\'ia de las arquitecturas elaboradas est\'an orientadas a groupwares con ambientes espec\'ificos, es decir, usan el contexto de una situaci\'on en particular, por ejemplo un sistema para  procesos empresariales, tomar\'ia en cuenta datos sobre el giro de la empresa y las actividades que se realizan dentro de la empresa inherentes a las pol\'iticas internas de la organizaci\'on; estos datos no ser\'ian de utilidad para otro tipo de situaciones como en un groupware para desarrollo de software que utiliza otro tipo de variables contextuales.

En el presente trabajo se hace una revisi\'on de las arquitecturas conscientes del contexto para poder dise\'nar una dirigida a sistemas groupware independientemente de el ambiente en el que se desempe\'nen, la arquitectura parte de un trabajo antes propuesto en el que se divide el manejo de la informaci\'on contextual en 3 fases: la recuperaci\'on de la informaci\'on, geesti\'on de los datos recuperados y el procesamiento y uso de el contexto obtenido. Inherente a estas tareas est\'an las de implementar una ontolog\'ia o clasificaci\'on de datos contextuales, t\'ecnicas de inferencia de datos contextuales y distribuci\'on de los resultados obtenidos. Una vez terminado el dise\'no se implementar\'a la arquitectura con un groupware y se har\'an pruebas.