\section{Introducci\'on}
En los trabajos colaborativos un grupo de personas unen sus esfuerzos para alcanzar un objetivo realizando tareas que en ocasiones dependen de otras tareas realizadas por otros integrantes del grupo. El trabajo colaborativo asistido por computadora\footnote{CSCW por sus siglas en ingl\'es(\textit{Comupter Supported Collaborative Work})} es el \'area que estudia los sistemas computacionales que ayudan a los grupos de trabajo a mejorar la coordinaci\'on, cooperaci\'on y comunicaci\'on que hay entre los integrantes del grupo de trabajo. Entre los sistemas que estudia el CSCW est\'an los Groupware, que son sistemas que fungen como medio para que los usuarios interact\'uen entre si y puedan llevar a cabo tareas en conjunto. Para facilitar la interacci\'on de estos usuarios con el sistema y con otros usuarios necesitan tener un grado aceptable de consciencia de la situaci\'on que los rodea a ellos como individuos, de los miembros del grupo del que forman parte y de las tareas que se est\'an llevando a cabo. 

Debido a esto surge la necesidad de dotar a los sistemas con el conocimiento necesario para poder proporcionar tales niveles de conciencia al usuario a partir de esta necesidad se desarrollan sistemas conscientes del contexto los cuales tienen conocimiento de la situaci\'on del ambiente en el que operan, as\'i, los Sistemas Groupware Conscientes del Contexto \footnote{CAGS por sus siglas en ingl\'es \textit{Context Aware Groupware Sistems}} tienen ambas caracter\'isticas: apoyan el trabajo colaborativo de un grupo de usuarios y tienen conocimiento del contexto que los rodea y opera en consecuencia a esta informaci\'on, para esto se crean arquitecturas o marcos de trabajo conscientes del contexto que puedan ser integradas a un Groupware y as\'i puedan funcionar como un CAGS.

El problema de estos sistemas es que est\'an basados en escenarios particulares satisfaciendo necesidades espec\'ificas del problema; por ejemplo, Meeting Reminder Agent\cite{anhalt2001toward}, toma el tiempo de distracci\'on de actividades, la ubicaci\'on, y el sonido en un campus para avisar al usuario de los lugares donde se est\'an llevando a cabo reuniones, y sugerir de acuerdo con los intereses del usuario, conferencias que se lleven a cabo. Otro caso es Portable Help Desk (PHD) \cite{anhalt2001toward} que es una aplicaci\'on que toma en cuenta la cercan\'ia de los miembros de un grupo y su disponibilidad para poder brindar apoyo a otros miembros y carecen de generalidad para poder ser aplicados en ambientes con un contexto diferente al que han sido desarrollados. Por \'ultimo CO2DE \cite{schmidt1992taking} que se concentra en la edici\'on colaborativa as\'incrona de diagramas y resoluci\'on de conflictos que mantiene el contexto individual de los integrantes del proyecto y evita traslapar ediciones manteniendo a cada contexto de los individuos en una rama diferente al proyecto original.

Se observa en los tres ejemplos que cada aplicaci\'on usa datos contextuales propios del escenario, y si se intenta integrar el modelo de uno de ellos a otro, har\'ia falta a\'nadir las caracter\'isticas que el anterior no ten\'ia, lo que dificulta el modelado de soluciones. Por lo tanto son requeridos servicios fundamentales de contexto gen\'erico para hacer de la conciencia del contexto una tecnolog\'ia factible que puede ser f\'acilmente incorporada a una variedad de software (Pascoe, J., Ryan, N., \& Morse, D., 1999).

En el presente trabajo se propone una arquitectura consciente del contexto, que parte de otro proyecto de investigaci\'on\cite{montane2013context}, constituida por tres niveles: recuperaci\'on de informaci\'on contextual, manejo y gesti\'on de datos y uso del contexto. Se hace especial \'enfasis en el uso de contexto y en la forma en que se aplican reglas de inferencia sobre un grupo de datos contextuales relacionados entre si para obtener un resultado, ya sea un comando ejecutable en el groupware o informaci\'on importante para la tarea del usuario.

%\begin{figure}[h!]
%  \centering
%    \includegraphics[scale=0.46]{images/arqui}
%  \caption{Comparaci\'on de arquitecturas que soportan consciencia del contexto\cite{montane2013context}\cite{dey1999architecture}\cite{kamoun2012fadyrcos}\cite{decouchant2013adapting}\cite{guermah2013ontology}}
%  \label{cmp:arc}
%\end{figure}

