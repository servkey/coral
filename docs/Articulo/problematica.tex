\section{Definici\'on del problema}
En un ambiente colaborativo existe mucha interacci\'on entre los integrantes, necesitan estar comunic\'andose  constantemente para poder coordinar sus esfuerzos, y en un groupware es a\'un m\'as dif\'icil ya que el sistema se vuelve intermediario entre los usuarios,  para poder hacer estas interacciones m\'as fluidas el sistema necesita saber c\'omo ayudar a los usuarios y esto se logra d\'andole datos del contexto en general y logrando que razone y act\'ue  en beneficio del grupo.

Para que una aplicaci\'on pueda ser consciente del contexto, debe de ser capaz de adquirir informaci\'on contextual, gestionarla, y procesarla para obtener resultados que permitan ejecutar un comando o mostrar informaci\'on para el usuario. 

Se han propuesto muchos modelos y arquitecturas para el desarrollo de sistemas consientes del contexto, el problema de estos sistemas es que est\'an basados en escenarios particulares, es decir, satisfaciendo necesidades espec\'ificas del problema; por ejemplo, Meeting Reminder Agent\cite{anhalt2001toward}, toma el tiempo de distracci\'on de actividades, la ubicaci\'on, y el sonido en un campus para avisar al usuario de los lugares donde se est\'an llevando a cabo reuniones, y sugerir de acuerdo con los intereses del usuario, conferencias que se lleven a cabo. Otro caso es Portable Help Desk (PHD) \cite{anhalt2001toward} que es una aplicaci\'on que toma en cuenta la cercan\'ia de los miembros de un grupo y su disponibilidad para poder brindar apoyo a otros miembros y carecen de generalidad para poder ser aplicados en ambientes con un contexto diferente al que han sido desarrollados. Por \'ultimo CO2DE \cite{schmidt1992taking} que se concentra en la edici\'on colaborativa as\'incrona de diagramas y resoluci\'on de conflictos que mantiene el contexto individual de los integrantes del proyecto y evita traslapar ediciones manteniendo a cada contexto de los individuos en una rama diferente al proyecto original.

Se puede observar en los tres ejemplos anteriores que cada aplicaci\'on usa elementos diferentes del contexto seg\'un las necesidades que cubren, y si se quiere integrar el modelo de uno de ellos a otro \'ambito diferente, har\'ia falta integrar las caracter\'isticas que el anterior no ten\'ia, lo que supone una mayor carga en el modelado de soluciones. Por lo tanto, servicios fundamentales de contexto gen\'erico son requeridos para hacer de la conciencia del contexto una tecnolog\'ia factible que puede ser f\'acilmente incorporada a una variedad de software (Pascoe, J., Ryan, N., \& Morse, D., 1999).
 
Montan\'e \cite{montane2013context} propone una arquitectura para apoyar el trabajo colaborativo en los groupware en cuyos m\'odulos se pueden encontrar los elementos anteriores. La primera capa es de adquisici\'on de datos contextuales en la que cada m\'odulo trabaja con tipos diferentes de datos contextuales; en la capa de gesti\'on de contexto se almacena la informaci\'on obtenida en una base de datos para f\'acil acceso, aqu\'i se guardar\'an, actualizar\'an y recuperar\'an los datos hist\'oricos de la aplicaci\'on; en la \'ultima capa, que es la de uso de contexto, se encuentran 2 m\'odulos, uno que procesa y razona la informaci\'on contextual, y otro que entrega los resultados al sistema groupware.

Actualmente se encuentra desarrollada la capa de adquisici\'on de datos, pero para poder implementar la arquitectura en su totalidad, hacen falta desarrollar un modelo de las capas de gesti\'on de datos y de razonamiento contextual, de ah\'i el surgimiento del presente trabajo.
