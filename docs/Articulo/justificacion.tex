\section{Justificaci\'on}
La comunicaci\'on, coordinaci\'on y cooperaci\'on entre grupos colaborativos de trabajo es muy importante, con su eficiencia aumenta el rendimiento de los usuarios en el trabajo que realizan, fomentando la productividad. Con los groupware se mejora la calidad de operaci\'on de estos grupos, y a\'nadiendo la conciencia del contexto a este tipo de sistema, la interacci\'on que los usuarios tienen con los dispositivos o aplicaciones se hace de forma m\'as natural y fluida, permiti\'endoles concentrarse en la tarea que est\'an haciendo en lugar de detalles de comunicaci\'on y evitando la carga de procesamiento de informaci\'on que el sistema har\'a por ellos. Con la arquitectura que se va a dise\'nar, el desarrollo se vuelve menos complejo, evitando que los desarrolladores empiecen desde cero un proyecto que incluya groupware consciente del contexto.