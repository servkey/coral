\section{Antecedentes}
El trabajo colaborativo asistido por computadora (CSCW) es un \'area de investigaci\'on dirigida al dise\'no de sistemas de aplicaci\'on para una categor\'ia espec\'ifica de trabajo
\cite{schmidt1992taking}. Algunos de los sistemas que estudia esta \'area son los Groupware, sistemas de magnitud organizacional, que permiten la colaboraci\'on, comunicaci\'on y coordinaci\'on de un grupo para alcanzar una meta. Estos \'ultimos tres conceptos (colaboraci\'on, comunicaci\'on y coordinaci\'on) son de suma importancia para el trabajo colaborativo, para que estos sean llevados a cabo de manera eficiente es necesario que los miembros del grupo tengan conciencia de la situaci\'on en muchos niveles \cite{gutwin1996supporting}. 

En un groupware es dif\'icil que los usuarios tengan conciencia completa de todo el espacio de trabajo en el que participan, por ejemplo, en un juego colaborativo \cite{montane2013context}, cuando los usuarios juegan en una misma habitaci\'on pueden comunicarse directamente con sus compa\~neros, ver sus expresiones, y percibir sus sentimientos adem\'as de poderse comunicar con ellos directamente; mientras que, al jugarlo en habitaciones distintas les es m\'as dif\'icil poder captar este tipo de se\~nales. Adem\'as de eso, para que el sistema pueda proveer medios de colaboraci\'on que permitan a los usuarios comunicarse eficientemente, ellos tienen que interactuar expl\'icitamente con los medios de comunicaci\'on que ofrece el sistema, provocando distracciones que afectan la concentraci\'on de los usuarios para lograr el objetivo principal. Para dar soluci\'on a este tipo de problemas, se proporciona al sistema informaci\'on contextual de los usuarios.
 
El c\'omputo consciente del contexto es un t\'ermino discutido por primera vez en el trabajo de Schilit y Theimer como software que se adapta de acuerdo al contexto, esto limita la definici\'on a aplicaciones que son informadas sobre el contexto y se adaptan a \'el\cite{schillit1994disseminating}. En investigaciones m\'as recientes tomando como referencia investigaciones anteriores, se define computaci\'on consciente del contexto como un sistema que usa el contexto para proporcionar informaci\'on relevante y/o servicios al usuario, donde la relevancia depende de la tarea del usuario\cite{dey2001conceptual}. La definici\'on anterior se puede ver reflejada un sistema gu\'ia de turistas, donde la informaci\'on dada por el sistema es de inter\'es para los usuarios y la actividad que realizan, como por ejemplo, notificar de eventos pr\'oximos, o recomendar actividades o lugares a los usuarios. En el caso de un juego de video colaborativo, el sistema puede elevar la dificultad dependiendo del nivel del usuario, as\'i como notificar a los usuarios de las actividades de sus compa\~eros.

Se entiende por contexto la situaci\'on actual que tiene lugar en una actividad realizada por un sujeto o un grupo de entidades. Estas situaciones est\'an definidas por diferentes elementos que responden a las preguntas ?`qui\'en?, ?`d\'onde?, ?`cu\'ando? y ?`qu\'e?. Schilit y Theimer \cite{schillit1994disseminating} se refieren al contexto como la ubicaci\'on, identidades de personas y objetos cercanos, y los cambios en esos objetos. Mientras que la definici\'on de Dey \cite{dey2001conceptual} menciona que el contexto es cualquier informaci\'on relevante sobre las entidades en la interacci\'on entre el usuario y una computadora, incluy\'endolos a ambos, una entidad puede ser una persona, lugar u objeto considerado relevante para la interacci\'on entre un usuario y una aplicaci\'on.
