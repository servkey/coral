\section{M\'etodo}
Para este trabajo se estiman las siguientes actividades:

Fase 1: Se revisar\'a el estado del arte de las arquitecturas para sistemas conscientes del contexto

Fase 2: Se har\'a una investigaci\'on de las propiedades y elementos de los sistemas colaborativos e indagar como se puede establecer una comunicaci\'on con ellos para poder extraer y enviar los datos contextuales y resultados del procesamiento

Fase 3: Clasificaci\'on de contexto de manera general, para empezar a planear la forma de procesarlo, de esto saldr\'a un modelo conceptual de contexto en el que se basar\'a el desarrollo de la arquitectura antes propuesta.

Fase 4: Distribuci\'on l\'ogica de m\'odulos de operaci\'on que participaran en el procesamiento de la informaci\'on contextual, producto de esto ser\'a un modelo de flujo de datos de los m\'odulos propuestos. Modelo de comunicaci\'on basado en servicios y se distribuci\'on de m\'odulos en distintos servidores para empezar a estructurar la arquitectura.

Fase 5: Codificaci\'on de m\'odulos para empezar a implementar la arquitectura ya propuesta, se probaran los m\'odulos por separado y en conjunto para determinar la efectividad del producto. Aplicaci\'on de la arquitectura a un sistema groupware para realizar un caso de estudio, el sistema es un juego FPS (first person shoter) de c\'odigo abierto, en particular AssaultCube. Documentaci\'on de los resultados y conclusi\'on.
