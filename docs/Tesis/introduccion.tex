\section{Introducci�n}
\label{sec:intro}
Actualmente la mayor\'ia de las actividades, operaciones o procedimientos que se llevan a cabo en la industria, el entretenimiento y la vida diaria son desarrollados de manera conjunta, un grupo de personas unen sus esfuerzos para poder realizar diversas tareas. Tradicionalmente los sistemas de informaci\'on son un medio importante en muchas de estas tareas. El \'area que estudia estos sistemas es el Trabajo Colaborativo Asistido por Computadora o CSCW por sus siglas en ingl\'es (Computer Supported Collaborative Work). Entre los sistemas que estudia CSCW est\'an, en particular, los groupware, sistemas que apoyan a un grupo de trabajo proporcionando comunicaci\'on e informaci\'on a los usuarios sobre la actividad que se est\'an realizando, Ellis \cite{ellis1991groupware} define Groupware como sistemas computacionales que apoyan a grupos de personas ocupadas en una tarea en com\'un(u objetivo) y que proporcionan una interfaz para un ambiente compartido. Para hacer que la interacci\'on humano-computadora se lleve a cabo de manera m\'as natural y transparente, se dota al Groupware con la habilidad de percibir y trabajar con datos que describan la situaci\'on que rodea al grupo, esto se le conoce como consciencia contextual, una caracter\'istica que trae consigo el surgimiento de los Sistemas Groupware Conscientes del Contexto (CAGS).


Para que los Groupware Conscientes del Contexto(CAGS por sus siglas en ingl\'es) puedan trabajar con datos contextuales es necesario tener una descripci\'on del ambiente en el que va a trabajar, hace falta especificar los elementos que se deben de tomar en cuenta, por ejemplo, para un sistema de edici\'on simultanea de textos se debe trabajar informaci\'on como las modificaciones que se han hecho, la fecha de las modificaciones, los permisos de los usuarios tienen para acceder al documento, etc. mientras que para un sistema de gu\'ia de turistas se toman elementos como la ubicaci\'on del usuario, el lugar(por ejemplo un edificio, o en un espacio abierto), fechas de eventos pr\'oximos, etc. Como se puede observar en ambos casos, se usan descripciones diferentes del ambiente del sistema y esto implica tener que desarrollar 2 sistemas completamente diferentes, uno para cada caso en espec\'ifico. Esto se vuelve un problema al momento de desarrollar varios sistemas que requieren el procesamiento de informaci\'on contextual, ya que hay que estar cambiando la especificaci\'on del contexto cada vez y, en el peor caso, desarrollar desde cero un sistema para un ambiente completamente distinto de los creados anteriormente.

Para evitar este problema, se propone modelar e implementar una arquitectura orientada a servicios para sistemas groupware conscientes del contexto, cuyo dise\'no tome en cuenta los aspectos contextuales m\'as generales, est\'a arquitectura deber\'a poder utilizada en cualquier \'ambito reduciendo as\'i tiempo de desarrollo, ser\'a probada en un juego colaborativo (un videojuego de disparos en primera persona) y se documentar\'an los resultados para futuras investigaciones.

