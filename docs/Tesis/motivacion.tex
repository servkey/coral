\section{Motivaci\'on}
% % % checar primer parrafo
La comunicaci\'on, coordinaci\'on y cooperaci\'on entre grupos colaborativos de trabajo es muy importante, con su eficiencia aumenta el rendimiento de los usuarios en el trabajo que realizan, fomentando la productividad. Con los groupware se mejora la calidad de operaci\'on de estos grupos, y a\'nadiendo la conciencia del contexto a este tipo de sistema, la interacci\'on que los usuarios tienen con los dispositivos o aplicaciones se hace de forma m\'as natural y fluida, permiti\'endoles concentrarse en la tarea que est\'an haciendo en lugar de detalles de comunicaci\'on y evitando la carga de procesamiento de informaci\'on que el sistema har\'a por ellos. Con la arquitectura que se va a dise\~nar, el desarrollo se vuelve menos complejo, evitando que los desarrolladores tengan que volver a programar otras  caracter\'isticas.
 
En un ambiente colaborativo los integrantes necesitan estar comunic\'andose  constantemente para poder coordinar sus esfuerzos. En un groupware esto implica que el sistema se vuelva intermediario entre los usuarios,  para poder hacer estas interacciones m\'as fluidas el sistema necesita saber c\'omo ayudar a los usuarios y esto se logra d\'andole datos del contexto en general y logrando que act\'ue  en beneficio del grupo. Para que una aplicaci\'on pueda ser consciente del contexto, debe de ser capaz de adquirir informaci\'on contextual, gestionarla, y usarla para obtener resultados que permitan ejecutar un comando o mostrar informaci\'on para el usuario. 

Anteriormente en el trabajo de Montan\'e\citep{montane2013context} se propuso una arquitectura para apoyar el trabajo colaborativo en los groupware que adem\'as eval\'ua la presencia social de sus usuarios para medir el nivel de colaboraci\'on, en cuyos m\'odulos se pueden encontrar los elementos anteriores. La primera capa es de adquisici\'on de datos contextuales en la que cada m\'odulo trabaja con tipos diferentes de datos contextuales; en la capa de gesti\'on de contexto se almacena la informaci\'on obtenida para acceder a ella m\'as tarde, aqu\'i se guardar\'an, actualizar\'an y recuperar\'an los datos hist\'oricos de la aplicaci\'on; en la \'ultima capa, que es la de uso de contexto, se encuentran dos m\'odulos, uno que procesa y razona la informaci\'on contextual, y otro que entrega los resultados al sistema groupware.

Se han propuesto modelos y arquitecturas para el desarrollo de sistemas consientes del contexto, sin embargo estos sistemas est\'an basados en escenarios particulares, es decir, satisfaciendo necesidades espec\'ificas de un dominio; por ejemplo, Meeting Reminder Agent\citep{anhalt2001toward}, toma el tiempo de distracci\'on de actividades, la ubicaci\'on, y el sonido en un campus para avisar al usuario de los lugares donde se est\'an llevando a cabo reuniones, y sugerir de acuerdo con los intereses del usuario, conferencias que se lleven a cabo. Otro caso es Portable Help Desk (PHD) \citep{anhalt2001toward} que es una aplicaci\'on que toma en cuenta la cercan\'ia de los miembros de un grupo y su disponibilidad para poder brindar apoyo a otros miembros y carecen de generalidad para poder ser aplicados en ambientes con un contexto diferente al que han sido desarrollados. Por otra parte CO2DE \citep{schmidt1992taking} que se concentra en la edici\'on colaborativa as\'incrona de diagramas y resoluci\'on de conflictos que mantiene el contexto individual de los integrantes del proyecto y evita traslapar ediciones manteniendo a cada contexto de los individuos en una rama diferente al proyecto original.

Las aplicaciones mencionadas anteriormente usan elementos diferentes del contexto seg\'un las necesidades que cubren, y es complicado re-usar los modelos propuestos en cada uno de ellos con otros sistemas. Por ello falta integrar caracter\'isticas que parcialmente son incluidos en los trabajos explorados, lo que supone una mayor carga en el modelado de soluciones. Por lo tanto, servicios fundamentales de contexto gen\'erico son requeridos para hacer de la conciencia del contexto una tecnolog\'ia factible que puede ser f\'acilmente incorporada a una variedad de software \citep{pascoe1999}. Para evitar este problema, se propone modelar e implementar una arquitectura orientada a servicios para sistemas groupware conscientes del contexto, cuyo dise\~no tome en cuenta los aspectos contextuales m\'as generales, que pueda utilizarse en otros \'ambitos, reduciendo as\'i tiempo de desarrollo. La arquitectura ser\'a probada con un juego colaborativo (un video juego de disparos en primera persona) y se documentar\'an los resultados para futuras investigaciones.
 
Actualmente se encuentra desarrollada la capa de adquisici\'on de datos y un m\'odulo para percibir la presencia social de los usuarios, pero aun carece de un mecanismo de razonamiento contextual.