% %Resumen % %
\section*{Resumen}
El trabajo colaborativo es una actividad en la que un grupo de personas une sus esfuerzos para alcanzar un objetivo realizando tareas en las que otras personas pueden estar involucradas. El Trabajo Colaborativo Asistido por Computadora (CSCW) \footnote{CSCW por sus siglas en ingl\'es(\textit{Comupter Supported Collaborative Work})} es el \'area que estudia los sistemas computacionales que ayudan a estos grupos de trabajo a mejorar la coordinaci\'on, cooperaci\'on y comunicaci\'on que hay entre los integrantes del grupo de trabajo.

Entre los sistemas que estudia el CSCW se encuentran los Groupware, sistemas que fungen como medio para que los usuarios interact\'uen entre s\'i y puedan llevar a cabo tareas en conjunto. Para facilitar la interacci\'on de estos usuarios con el sistema y con otros usuarios, los groupware les proporcionan comunicaci\'on e informaci\'on para que tengan conocimiento de la situaci\'on actual con el fin de facilitar la toma de decisiones. As\'i surge la necesidad de dotar a los sistemas con el conocimiento necesario para poder proporcionar tales niveles de conscienca al usuario, a partir de esta necesidad se desarrollan sistemas conscientes del contexto. Los Sistemas Groupware Conscientes del Contexto (CAGS por sus siglas en ingl\'es)\footnote{ \textit{Context Aware Groupware Systems}} tienen ambas caracter\'isticas antes descritas: apoyan el trabajo colaborativo de un grupo de usuarios y soportan el uso de contexto y opera en consecuencia a esta informaci\'on, para eso se dise\~an arquitecturas conscientes del contexto que se integran a un groupware y puedan, entre ambos sistemas, funcionar como un CAGS. 

En el presente trabajo se hace una revisi\'on de las arquitecturas conscientes del contexto para analizar sus caracter\'isticas y poder dise\~nar una dirigida a sistemas groupware independiente del escenario en el que se desempe\~nen, la arquitectura parte de un trabajo antes propuesto\cite{montane2013context} en el que se divide el manejo de la informaci\'on contextual en 3 etapas: la recuperaci\'on de la informaci\'on, gesti\'on de los datos recuperados y el uso y distribuci\'on de resultados. Para la generaci\'on de CAGS es necesario un mecanismo de razonamiento contextual, el cual pueda generar resultados a partir de informaci\'on del contexto del sistema, para esto se propone un lenguaje de definici\'on de comportamiento del que se infieren dichos resultados, adem\'as se necesita implementar una ontolog\'ia o clasificaci\'on contextual que pueda ser reutilizada en cualquier caso de estudio. Una vez terminado el dise\~no se implementar\'a la arquitectura con un groupware y se har\'an pruebas.

Actualmente la mayor\'ia de las arquitecturas elaboradas est\'an orientadas a groupwares con ambientes espec\'ificos, es decir, usan el contexto de un escenario en particular, por ejemplo un sistema para  procesos empresariales, tomar\'ia en cuenta datos sobre el giro de la empresa y las actividades que se realizan internamente inherentes a las pol\'iticas de la organizaci\'on; estos datos no ser\'ian de utilidad para otro tipo de situaciones como en un groupware para desarrollo de software que se centra en los proyectos de desarrollo y que utiliza otro tipo de variables contextuales.