En la actualidad existen actividades o procedimientos que se llevan a cabo en la vida cotidiana que requieren ser desarrolladas de manera conjunta, un grupo de personas que unen sus esfuerzos para realizar diversas tareas. En ocasiones, algunas de estas tareas son soportadas por los sistemas de informaci\'on siendo un medio por el cual los usuarios interactuan para realizar su cometido. El \'area que estudia este tipo de sistemas es el Trabajo Colaborativo Asistido por Computadora o CSCW por sus siglas en ingl\'es (\textit{Computer Supported Collaborative Work}), entre los sistemas que estudia el CSCW est\'an los Groupware, sistemas de magnitud organizacional que apoyan a un grupo de trabajo permitiendo la colaboraci\'on, comunicaci\'on y coordinaci\'on de un grupo para alcanzar una meta, Ellis\citep{ellis1991groupware} define Groupware como sistemas computacionales que apoyan a grupos de personas ocupadas en una actividad(u objetivo) en com\'un y que proporcionan una interfaz en un ambiente compartido.

Para hacer que la interacci\'on humano-computadora se lleve a cabo de manera natural y transparente, se dota al Groupware con la caracter\'istica de percibir y trabajar con datos que describan la situaci\'on que rodea al grupo, a esto se le conoce como conciencia contextual, una caracter\'istica que trae consigo el surgimiento de los Sistemas Groupware Conscientes del Contexto (CAGS).

Se entiende por contexto la situaci\'on actual que tiene lugar en una actividad realizada por un sujeto o un grupo de entidades. Estas situaciones est\'an definidas por diferentes elementos que responden a las preguntas ?`Qui\'en?, ?`D\'onde?, ?`Cu\'ando? y ?`Qu\'e?. Schilit y Theimer \citep{schillit1994disseminating} se refieren al contexto como la ubicaci\'on, identidades de personas y objetos cercanos, y los cambios en esos objetos. Mientras que la definici\'on de Dey \citep{dey2001conceptual} menciona que el contexto es cualquier informaci\'on relevante sobre las entidades en la interacci\'on entre el usuario y una computadora, incluy\'endolos a ambos, una entidad puede ser una persona, lugar u objeto considerado relevante para la interacci\'on entre un usuario y una aplicaci\'on.

El c\'omputo consciente del contexto es un t\'ermino discutido por primera vez en el trabajo de Schilit y Theimer como software que se adapta de acuerdo al contexto, esto limita la definici\'on a aplicaciones que son informadas sobre el contexto y se adaptan a \'el\citep{schillit1994disseminating}. En investigaciones m\'as recientes se define computaci\'on consciente del contexto como un sistema que usa el contexto para proporcionar informaci\'on relevante y/o servicios al usuario, donde la relevancia depende de la tarea del usuario\citep{dey2001conceptual}. La definici\'on anterior se puede ver reflejada un sistema gu\'ia de turistas, donde la informaci\'on dada por el sistema es de inter\'es para los usuarios y la actividad que realizan, como por ejemplo, notificar de eventos pr\'oximos, o recomendar actividades o lugares a los usuarios. En el caso de un juego de video colaborativo, el sistema puede elevar la dificultad dependiendo del nivel del usuario, as\'i como notificar a los usuarios de las actividades de sus compa\~neros.

Para que los Groupware Conscientes del Contexto(CAGS por sus siglas en ingl\'es) puedan trabajar con datos contextuales es necesario tener una descripci\'on del ambiente en el que va a trabajar, hace falta especificar los elementos que se deben de tomar en cuenta, por ejemplo, para un sistema de edici\'on simultanea de textos se debe trabajar informaci\'on como las modificaciones que se han hecho, la fecha de las modificaciones, los permisos de los usuarios tienen para acceder al documento, etc. mientras que para un sistema de gu\'ia de turistas se toman elementos como la ubicaci\'on del usuario, el lugar(por ejemplo un edificio, o en un espacio abierto), fechas de eventos pr\'oximos, etc. Como se puede observar en ambos casos, se usan descripciones diferentes del ambiente del sistema y esto implica tener que desarrollar dos sistemas completamente diferentes, uno para cada caso en espec\'ifico. Esto se vuelve un problema al momento de desarrollar varios sistemas que requieren el procesamiento de informaci\'on contextual, ya que hay que cambiar la especificaci\'on del contexto cada vez y, en el peor caso, desarrollar desde cero el sistema.

De modo que los usuarios puedan coordinarse, comunicarse y colaborar de forma eficiente, es necesario que los miembros del equipo tengan un alto nivel de conciencia del espacio de trabajo en el que participan, esto se cumple cuando los usuarios se encuentran en el mismo lugar grogr\'afico, ya que pueden interactuar en tiempo real y as\'i mantenerse informados de la actividad de los dem\'as, por ejemplo, en un juego colaborativo \citep{montane2013context}, cuando los usuarios juegan en una misma habitaci\'on pueden comunicarse directamente con sus compa\~neros, ver sus expresiones; al jugarlo en habitaciones distintas les es m\'as dif\'icil poder captar este tipo de se\~nales. Adem\'as de eso, para que el sistema pueda proveer medios de colaboraci\'on que permitan a los usuarios comunicarse eficientemente, ellos tienen que interactuar expl\'icitamente con los medios de comunicaci\'on que ofrece el sistema, provocando distracciones que afectan la concentraci\'on de los usuarios para lograr el objetivo principal. Para dar soluci\'on a este tipo de problemas, se proporciona al sistema informaci\'on contextual de los usuarios.